\documentclass{article}
\usepackage{mathtools, amssymb, amsthm} % imports amsmath
\usepackage{fancyhdr}

\pagestyle{fancy}
\fancyhf{}
\lhead{More Quantifiers and Logical Equivalences}
\rhead{Lecture 4}

\begin{document}

\section{Uniqueness Quantifier $\exists$! $\exists_{1}$}
\fbox{\strut There exists a unique x that p(x) is true, there is exactly 1, there is one and only one.}\\
\\
If p(x) denotes "$x+1=0$ and x is an integer, then $\exists$!xP(x) is true because only one unique x can make the equation true.\\
If p(x) denotes "$x > 0$, then $\exists$!xP(x) is false because there are many x values that can make this true.

\section{Quantifiers with Restricted Domains}
\begin{enumerate}
    \item
        \begin{displaymath}
        \forall x < 0, x^{2} > 0 \to \forall x(x<0\to x^{2}>0)
        \end{displaymath}

    \item
        \begin{displaymath}
        \forall y \neq 0, y^{3} \neq 0 \to \forall y(y\neq 0 \to y^{3}\neq 0)
        \end{displaymath}

    \item
        \begin{displaymath}
        \exists z > 0, z^{2} = 2 \to \exists z(z > 0 \wedge z^{2}=2)
        \end{displaymath}
\end{enumerate}

\section{Precedence of Quantifiers}
$\forall$ and $\exists$ have higher precedence than all logical operators.\\

\begin{displaymath}
\forall x p(x)\vee q(x) \equiv (\forall x p(x))\vee q(x) \text{ instead of } \forall x (p(x)\vee q(x))
\end{displaymath}

\section{Binding Variables}
\fbox{
    \parbox{\textwidth}{
        When a quantifier is used on the variable x, this occurence of the variable is bound.
        If it is not bound, then it is free.
    }
}
\\
\\
The variables that occur in a propositional function of a predicate calculus must be bound or set to a particular value to turn to a proposition.\\
The part of a logical expression where the quantifier is applied = the scope of the quantifier.

TODO Add Examples...

\pagebreak
\section{Translating from English to Logic}
\fbox{
    \strut Translate to predicate logic: "Every student in this class has taken a course in Java."
}
\\
\\
\textbf{1. Decide on domain U.}
\begin{enumerate}
    \item
    U is all students in the class:\\
    J(x) will denote "x has taken a course in Java" $\to \forall xJ(x)$ (For all students in this class, x student has taken a course in Java).\\

    \item
    U is all people:\\
    S(x) denotes "x is a student in the class" and J(x) denotes "x has taken a course in Java".\\
    Therefore, it translates to $\forall x(S(x)\to J(x))$ (For all people who are a student in this class, x student has taken a course in Java.)\\
\end{enumerate}

\fbox{
    \strut Translate to predicate logic: "Some students in this class has taken a course in Java."
}
\\
\\
\textbf{1. Decide on domain U.}
\begin{enumerate}
    \item
    U is all the students in the class:\\
    $\exists xJ(x) \to$ There exists x student in this class that has taken a course in Java.
    \item
    U is all people:\\
    $\exists x(S(x)\wedge J(x)) \to$ There exists x person who is a student in this class and has taken a course in Java.\\
\end{enumerate}

\fbox{
    \strut Every student in this class has studied calculus.
}
\\
\\
1. Introduce x into the statement.\\
\indent For every student x in this class, x has studied calculus.\\
2. Let c(x) represent "x (student) has studied calculus".\\
\indent $\forall xc(x)$\\
3. Let s(x) represent "x (all people) is a student in this class".\\
\indent $\forall xs(x)$\\
4. $\forall xs(x) \to c(x) \to$ For every x person, x is a student in this class, thus has studied calculus.\\
5. $\forall xs(x)\wedge c(x) \to$ For every x person, x is a student in this class and has studied calculus.\\

\pagebreak
\section{Logical Equivalences}
$S\equiv T$: Statements S and T involving predicates and quantifiers are logically equivalent.\\
\indent If and only if they have the same truth value no matter what is substituted into these statements and the domain used for variables.\\


\fbox{
    \parbox{\textwidth}{
        \begin{center}
        $\forall x(p(x)\wedge q(x))\equiv x p(x)\wedge \forall xq(x)$\\
        $\forall x(p(x)\wedge q(x))\equiv \forall xp(x)\wedge \forall xq(x)$
        \end{center}
    }
}
\\
\\
These two statements must take the same truth value no matter what p and q is and no matter what domain is used.\\

\section{Negating Quantified Expressions}
\begin{center}
\begin{tabular}{ |p{3cm}|p{3cm}|p{3cm}|p{3cm}| }
    \hline
    Negation & Equivalent Statement & When is Negation True? & When is Negation False?\\
    \hline
    $\neg \exists xP(x)$ & $\forall x\neg P(x)$ & For every x, P(x) is false. & There is an x for which P(x) is true.\\
    \hline
    $\neg \forall xP(x)$ & $\exists x\neg P(x)$ & There is an x for which P(x) is false. & P(x) is true for every x.\\
    \hline
\end{tabular}
\end{center}


\end{document}