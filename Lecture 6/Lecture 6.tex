\documentclass{article}
\usepackage{mathtools, amssymb, amsthm} % imports amsmath
\usepackage{fancyhdr}

\pagestyle{fancy}
\fancyhf{}
\lhead{Inferences}
\rhead{Lecture 6}

\begin{document}

\section{Rules of Inference}

\begin{enumerate}
\item
Argument: A sequence of statements with a conclusion\\
An argument is considered valid if and only if it is impossible for the premises to be true and the conclusion is false.\\
Use rules of inference to deduce new statements from statements we have.

\item
Valid: The conclusion of the argument must follow the truth of proceeding statements/premise of the argument.\\
Example:

\fbox{
    \parbox{\textwidth}{
        Premises: If you have a correct password, then log onto the network.\\You have a correct password $\to$ You can log onto the network.
        }
    }
\item
Proof: Valid arguments that establish the truth of a mathematical statement.
\end{enumerate}


\section{Valid Arguments in Propositional Logic}

TODO

\section{Rules of Inference for Propositional Logic}

TODO

\section{Rules of Inference for Propositional Logic: Modus Ponens}

TODO

\section{Modus Tollens}

TODO

\section{Hypothetical Syllogism}

TODO

\section{Disjunctive Syllogism}

TODO

\section{Addition}

TODO

\section{Simplification}

TODO

\section{Conjunction}
TODO

\section{Resolution}

TODO

\section{Valid Arguments}
TODO

\section{Fallacies}
TODO

\section{Inference with Quantified Statements}
TODO

\section{Using Rules of Inference}
TODO

\section{Universal Modus Ponens}
TODO

\section{Universal Modus Tollens}
TODO

\end{document}