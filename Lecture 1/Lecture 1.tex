\documentclass{article}
\usepackage{mathtools, amssymb, amsthm} % imports amsmath
\usepackage{fancyhdr}

\pagestyle{fancy}
\fancyhf{}
\lhead{What is Discrete Mathematics?}
\rhead{Lecture 1}

\begin{document}

\section{Logical Operators:}

\begin{enumerate}
    \item
    $\neg$ Negation
    
    \item
    $\wedge$ Conjunction\\(p$\wedge$q) when both p and q are true.

    \item
    $\vee$ Disjunction\\(p$\vee$q) is false when both p and q are false.

    \item
    $\oplus$ Exclusive\\(p$\oplus$q) is true when exactly 1 of p, q is true.

    \item
    $\to$ Conditional\\(p$\to$q) is false when p is true and q is false.\\
    \\The correct way to think about this is: p only if q. This means that p is true if and only if q.\\
    The incorrect way is: q unless $\neg$p. This means that q is true unless p is false.\\
    But, (p$\to$q) is true whenever q is true. Hence, q unless $\neg$p is not logically equivalent.\\

    \item
    $\leftrightarrow$ Biconditional\\(p$\leftrightarrow$q) is p if and only if q.


\end{enumerate}

\section{Tables}
\begin{center}
\begin{tabular}{||c c||}
    \hline
    Precedence & Operator\\
    \hline
    1 & $\neg$\\
    \hline
    2 & $\wedge$\\
    \hline
    3 & $\vee$\\
    \hline 
    4 & $\to$\\
    \hline
    5 & $\leftrightarrow$\\
    \hline

\end{tabular}
\end{center}

\section{Converse, Contrapositive, and Inverse}
For (p$\to$q):
\begin{enumerate}
    \item
    Converse: q $\to$ p

    \item
    Contrapositive: $\neg$q $\to$ $\neg$p\\This is equivalent to conditional statements.

    \item
    Inverse: $\neg$p $\to$ $\neg$q
\end{enumerate}

\end{document}