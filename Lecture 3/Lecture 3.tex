\documentclass{article}
\usepackage{mathtools, amssymb, amsthm} % imports amsmath
\usepackage{fancyhdr}

\pagestyle{fancy}
\fancyhf{}
\lhead{Predicates and Quantifiers}
\rhead{Lecture 3}

\begin{document}
\section{Predicate Logic}
\fbox{\strut Predicates are statements involving variables (x $>$ 3, "Student x is eating lunch.")}\\

\textbf{x $>$ 3}
\\x is considered the subject of the statement.\\
x, or the Predicate, is greater than 3 (of a property.)\\
p(x) can be called the value of the propositional function p at x.\\
\\Once the value is assigned to x, p(x) becomes a proposition and has a truth value.\\

\textbf{p(x) for x $>$ 3}
\begin{center}
    p(4): sets x = 4. 4 $>$ 3, which is true.\\
    p(2): sets x = 2. 2 $>$ 3, which is false.
\end{center}

\section{N-ary Predicate}
It is a statement with n variables.\\$p(x_{1}, x_{2},...,x_{n})$, which is the value of the propositional function p at the nth tuple.

\section{Quantifiers}
\fbox{\strut Keywords: all, some, many, none, few}\\

There are two types:
\begin{enumerate}
    \item
    Universal $\forall$xp(x): A predicate is true for every element under consideration.\\
    \\p(x) for all values of x in the domain, for all x p(x), or for every x p(x)\\
    \\This statement is false if and only if p(x) is not always true. The element that makes p(x) false 
    is a counterexample.

    \item
    Existential $\exists$xp(x): A predicate is true for there is one or more elements under consideration.\\
    \\There exists an element x in the domain such that p(x) = true. Also can be "for some, for at least one,
    there exists, there is."\\There is an x such that p(x), there is at least one x such that p(x), for some x p(x)
\end{enumerate}
\end{document}