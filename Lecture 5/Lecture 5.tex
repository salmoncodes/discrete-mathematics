\documentclass{article}
\usepackage{mathtools, amssymb, amsthm} % imports amsmath
\usepackage{fancyhdr}

\pagestyle{fancy}
\fancyhf{}
\lhead{Quantifiers, Loops, and Order of Quantification}
\rhead{Lecture 5}

\begin{document}

\section{Nested Quantifiers}

Let the domain be all real numbers:\\
\\
$\forall x\exists y(x+y)=0$\\
This is the same as:\\
$\forall xq(x)$ where q(x) is $\exists yp(x,y)$ and p(x,y) is x+y=0\\
\\
$\forall x\forall y(x + y = y + x)$\\
$\forall x\forall y\forall z(x + (y + z) = (x + y) + z)$\\
$\forall x\forall y((x > 0)\wedge(y<0)\to (xy<0))$

\section{Quantification as Loop}
\begin{enumerate}
    \item
    $\forall x\forall yp(x,y) \to$ For every x, for every y\\
    \\
    While looping through x, each x loops through y\\
    If p(x,y) is true for all x and y, then the statement is true.\\
    However, if x for p(x, y) is false, then the statement is false.\\

    \item
    $\forall x\exists yp(x,y) \to$ For every x, y exists\\
    \\
    Loop through x until a y makes p(x,y) true.\\
    If for every x there is a y that makes p(x,y) true, then the statement is true.\\
    However, if some x's doesn't have a y that makes p(x,y) true, then the statement is false.\\

    \item
    $\exists x\forall yp(x,y) \to$ There exists an x, for every y\\
    \\
    Loop through x until there is an x for which p(x,y) is always true when we loop through all values for y.\\
    It is true if there is an x that meets this criteria. False if there is no x.\\

    \item
    $\exists x\exists yp(x,y) \to$ For an x that exists, y exists\\
    \\
    Loop through x for where each x, loop through y until there is an x where y makes p(x,y) is true.\\
    This is false if there is no x with a y that makes p(x,y) true.\\
\end{enumerate}

\section{Order of Quantification}

\fbox{
    \strut Let p(x,y) be "x + y = y + x" and the domain is all real numbers.
}
\\
\\
Is the statement "$\forall x\forall yp(x,y) \equiv \forall y\forall xp(x,y)$" true?\\
\\
$\forall x\forall yp(x,y) \to$ For all real numbers of x, for all real numbers of y, x + y = y + x.\\
Since p(x,y) is true for all numbers x and y, $\forall x\forall yp(x,y)$ is true.\\
\\
$\forall y\forall xp(x,y) \to$ For all real numbers of y, for all real numbers of x, x + y = y + x.\\
This has the same meaning as the statement above, therefore the statements are equivalent.\\


\section{Translating Mathematical Statements}

\fbox{
    \strut The sum of two positive integers is always positive.
}
\\
\\
Translating this would be:\\
\\
$\forall x\forall y((x > 0 \wedge y > 0) \to (x +y > 0))$ and the domain is all integers for both variables.\\
$\forall x\forall y(x+y>0) \to$ For every x that is a positive integer, adding x to every y that is a positive integer will always be greater than 0.\\


\section{Translating Statements into English}

\begin{center}
    $\forall x(c(x)\vee \exists y(c(y)\wedge f(x,y)))$
\end{center}

\begin{enumerate}
    \item
    c(x) = "x has a computer", f(x,y) = "x and y are friends", domain = all students at the school\\
    
    \item
    For every x student in the school, x student has a computer or there exists y student with a computer and is friends with x student.
\end{enumerate}

\pagebreak
\section{Negating Nested Quantifiers}

\begin{center}

    $\neg \forall x\exists y(xy = 1)$
\\
    $\equiv \exists x\neg \exists y(xy = 1)$
\\
    $\equiv \exists x\forall y\neg (xy=1)$
\\
    $\equiv \exists x\forall y(xy\neq 1)$
\end{center}



\end{document}